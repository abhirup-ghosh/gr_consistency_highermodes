\documentclass[prd,preprintnumbers,twocolumn,eqsecnum,floatfix,a4paper,nofootinbib,superscriptaddress]{revtex4}
\usepackage{color}
\usepackage{calc}
\usepackage{amsmath,amssymb,graphicx}
\usepackage{amssymb,amsmath}
\usepackage{bm}
\usepackage{microtype}
\usepackage{booktabs}
\usepackage{times}
\usepackage[varg]{txfonts}
\usepackage[colorlinks, pdfborder={0 0 0}]{hyperref}
\usepackage[utf8]{inputenc}
\definecolor{LinkColor}{rgb}{0.75, 0, 0}
\definecolor{CiteColor}{rgb}{0, 0.5, 0.5}
\definecolor{UrlColor}{rgb}{0, 0, 0.75}
\hypersetup{linkcolor=LinkColor}
\hypersetup{citecolor=CiteColor}
\hypersetup{urlcolor=UrlColor}
\maxdeadcycles=1000
\allowdisplaybreaks
\textwidth 7 in
\hoffset -0.1in
\textheight 10in
\DeclareFontFamily{OT1}{pzc}{}
\DeclareFontShape{OT1}{pzc}{m}{it}{<-> s * [1.10] pzcmi7t}{}
\DeclareMathAlphabet{\mathpzc}{OT1}{pzc}{m}{it}
\newcommand{\comment}[1]{\textcolor{blue}{\textit{#1}}}
\newcommand{\ajith}[1]{\textcolor{red}{\textit{Ajith:#1}}}
\newcommand{\checkthis}{\textcolor{magenta}{(CHECKTHIS)}}
\newcommand{\vijay}[1]{\textcolor{cyan}{Vijay: #1}}
\newcommand{\io}{\iota}
\newcommand{\p}{\phi}
\newcommand{\vp}{\varphi}

\newcommand{\h}{\mathpzc{h}}
\newcommand{\Hhat}{\hat{\mathpzc{H}}}
\newcommand{\B}{\mathpzc{B}}
\newcommand{\hlm}{\mathpzc{h}_{\ell m}}
\newcommand{\xilm}{\xi_{\ell m}}
\newcommand{\Ylm}{{Y}^{-2}_{\ell m}}
\newcommand{\Y}{{Y}^{-2}}
\newcommand{\hc}{h_\times}
\newcommand{\hp}{h_+}
\newcommand{\Fc}{F_\times}
\newcommand{\Fp}{F_+}
\newcommand{\Mf}{M_f}
\newcommand{\cA}{\mathpzc{A}}
\newcommand{\lm}{_{\ell m}}
\newcommand{\deff}{d_\mathrm{eff}}
\newcommand{\rmi}{\mathrm{i}}
\newcommand{\blambda}{\bm{\lambda}}
\newcommand{\btheta}{\bm{\theta}}
\newcommand{\Mo}{M_{\odot}}
\newcommand{\FFe}{\mathrm{FF}_\mathrm{eff}}
\newcommand{\FF}{\mathrm{FF}}
\newcommand{\e}{\mathrm{e}}
\newcommand{\rhoopt}{\rho_\mathrm{opt}}
\newcommand{\rhosubopt}{\rho_\mathrm{subopt}}
\newcommand{\fqnm}{f}
\newcommand{\sigmaqnm}{\sigma}

\newcommand*{\skymapscale}{0.5}
\newcommand*{\paramestscale}{0.455}

\begin{document}

\newcommand{\be}{\begin{equation}}
\newcommand{\ee}{\end{equation}}
\newcommand{\ber}{\begin{eqnarray}}
\newcommand{\eer}{\end{eqnarray}}
\def\bea{\begin{eqnarray}}
\def\eea{\end{eqnarray}}
\newcommand{\etal}{\emph{et al}}

\title{A consistency test of general relativity using different multipoles of \\gravitational radiation from binary black holes}
\author{Siddharth Dhanpal}
\affiliation{International Centre for Theoretical Sciences, Tata Institute of Fundamental Research, Bangalore 560012, India}
\author{Abhirup Ghosh}
\affiliation{International Centre for Theoretical Sciences, Tata Institute of Fundamental Research, Bangalore 560012, India}
\author{Parameswaran~Ajith}
\affiliation{International Centre for Theoretical Sciences, Tata Institute of Fundamental Research, Bangalore 560012, India}
\affiliation{Canadian Institute for Advanced Research, CIFAR Azrieli Global Scholar, MaRS Centre, West Tower, 661 University Ave., Suite 505, Toronto, ON M5G 1M1, Canada}
\author{B.~S.~Sathyaprakash}
\affiliation{Penn State University}

\begin{abstract}
\end{abstract}
\preprint{LIGO-}
\maketitle
%%%%%%%%%%%%%%%%%%%%%%%%%%%%%%%%%%%%%%%%%%%%%%%%%%%%%%%%%%%%%%%%%%%%%%%%%%%%%%%%%%%%%%%%%%%%%%%%%%%%%%%%%%%%%%%%%%%%%%%%%%%%%%%%%%%%%%%%%%%%%%%`
\section{Introduction}
\section{Testing the consistency between different multipoles of the gravitational radiation}
\subsection{Method}\label{ssec:Method}
A gravitational wave (GW) $\mathrm{h}(t)$, can be decomposed into its two independent polarisations $(\mathrm{h} _+,\mathrm{h} _{\times})$ as: $\mathrm{h}(t) = \mathrm{h} _+(t) - i \mathrm{h} _{\times}(t)$. It can also be expanded in a basis of spin $-2$ weighted spherical harmonics, as:

\begin{equation}
\mathrm{h}(t; \iota, \phi_0, \lambda) = \sum _{l=2}^{\infty} \sum _{m=-l}^{l} \mathcal{Y}_{lm}^{-2} (\iota, \phi_0)\mathrm{h}_{lm}(t, \lambda)
\label{eq:spherical_harmonics}
\end{equation}
where $(\iota, \phi_0)$ define the direction of radiation in the source frame, and $\lambda$ defines the complimentary set of parameters required for waveform generation. These are the chirp mass $M_c$, asymmetric mass ratio $q$, luminosity distance $d_L$ and time of arrival $t_0$. Eq. ~\ref{eq:spherical_harmonics} can be rewritten to split the contributions from the dominant $(2,\pm 2)$ mode of gravitational radiation, and the sub-dominant or higher-order multipole moments:

\begin{equation}
\mathrm{h}(t; \iota, \phi_0, \lambda) = \mathcal{Y}_{2,\pm 2}^{-2} (\iota, \phi_0)\mathrm{h}_{2,\pm 2}(t, \lambda) + \sum _{\text{H.O.M}} \mathcal{Y}_{lm}^{-2} (\iota, \phi_0)\mathrm{h}_{lm}(t, \lambda)
\label{eq:test_HM}
\end{equation}
where the subscript H.O.M underneath the summation in the second term on the RHS indicates contribution from just the higher-order multipole moments of the gravitational radiation.

In this paper, we develop a test of general relativity (GR) which checks for consistency between estimates of $\lambda$, independently measured using the dominant and the sub-dominant modes of the gravitational waveform respectively. We redefine eq.~\ref{eq:test_HM} as:

\begin{equation}
\mathrm{h}(t; \iota, \phi_0, \lambda, \lambda') = \mathcal{Y}_{2,\pm 2}^{-2} (\iota, \phi_0)\mathrm{h}_{2,\pm 2}(t, \lambda) + \sum _{\text{H.O.M}} \mathcal{Y}_{lm}^{-2} (\iota, \phi_0)\mathrm{h}_{lm}(t, \lambda')
\end{equation}
where $\lambda$ are estimates from the dominant $(2,\pm 2)$ mode, and $\lambda'$ are estimates from the higher-order modes. If GR is correct, then these estimates should be consistent with each other. Alternatively, we define parameters describing the difference in the two independent estimates as:

\begin{equation}
\lambda ' = \lambda + \Delta \lambda
\end{equation}
By definition, if GR is correct, the posterior PDF on $\Delta \lambda$ should be consistent with zero.

For the remainder of our analysis we use the higher-mode model described in Mehta et al, 2017 [\textcolor{red}{Description of model in Mehta et al, 2017}].


\subsection{Analysis pipeline}
A GW detector records changes in the differential arm-length (DARM) between the two arms of the interferometer or \emph{strain} $d(t)$, due to noise $n(t)$, or the occasional passage of a GW, $h(t)$. Assuming an additive model for the strain:

\begin{equation}
d(t) = n(t) + h(t)
\label{eq:detector_strain}
\end{equation}
The noise $n(t)$ is assumed to describe a Gaussian random process with a given power spectral density (PSD), $S_n(f)$. The strain $h(t)$ is a linear combination of the two independent polarisations of GWs:

\begin{equation}
h(t) = F_+(\theta, \phi, \psi)\mathrm{h}_+(t) + F_{\times}(\theta, \phi, \psi)\mathrm{h}_{\times}(t)
\end{equation}
where $(F_+, F_x)$ are the antenna pattern functions of the GW detector, and $(\theta, \phi, \psi)$ define the sky position and polarisation of the GW source respectively. Assuming a GW signal from the coalescence of a binary black hole (BBH) system in a quasi-circular orbit, $h(t)$ can be completely described by a 15-dimensional parameter space $\theta$ in GR. The challenge of parameter estimation lies in inferring this paramater set $\theta$, given data $d$ and assuming a particular model of the waveform as our hypothesis $H$. The waveform model considered here is described in section.~\ref{ssec:Method}. One can then use the Bayes' theorem to obtain the posterior probability density function (PDF) on $\theta$ as:

\begin{equation}
p(\theta|d, H, I) = \frac{P(\theta|H, I) \mathcal{L}(d|\theta, H, I)}{E(d|H, I)}
\label{eq:Bayes_theorem}
\end{equation} 
The first term of the numerator on the RHS, $P(\theta|H,I)$ is the \emph{prior} PDF, the second term $\mathcal{L}(d|\theta, H,I)$ is the \emph{likelihood} function, and the term in the denominator $E(d|\theta, H,I)$ is a normalisation constant, called the \emph{evidence}. \emph{I} is any other information used in the analysis. The computation of the likelihood function is found to be less expensive in the Fourier domain, and hence, assuming Gaussian noise with a PSD given by $S_n(f)$, can be defined as:

\begin{equation}
\mathcal{L}(\tilde{d}(f)|\theta, H,I) = \text{exp}\Big[ -\frac{1}{2}\int_{f_{low}}^{f_{high}} \frac{|\tilde{d}(f) - \tilde{h}(f;\theta, H)|^2}{S_n(f)}df\Big]
\end{equation}
where $f_{low}$ and $f_{high}$ define the sensitivity bandwidth of the detector, and $(\tilde{d}(f), \tilde{h}(f))$ are the Fourier transforms of $(d(t), h(t))$ respectively. It is assumed that the detector response outside of this bandwidth is zero. 

Using the above definition for the likelihood function, one proceeds to estimate $\theta$ by stochastically sampling over the entire parameter space using  \emph{emcee}~\cite{goodman2010ensemble,foreman2013emcee}, an ensemble Markhov chain Monte Carlo sampler, which uses the underlying property of affine invariance of all Gaussian distributions to sample from highly skewed distributions a lot faster than standard single-particle methods, such as Metropolis-Hastings implementations. A MCMC scheme is a random walk through the parameter space $\theta$ generating samples of $\theta$ with a probability density which ultimately converges to the stationary distribution of the Markhov chain, the posterior PDF $p(\theta|d, H, I)$, using the equation of detailed balance. An \emph{ensemble} MCMC sampler implements a coordinated random walk of multiple "walkers" through the parameter space, such that each step of the Markhov chain, or updating the position of any one walker at a paricular time step, is influenced by the positions of the rest of the walkers. The specific move used by the sampler is called the \emph{stretch} move (described in detail in section 2. of ~\cite{goodman2010ensemble}). The algorithm requires the hand-tuning of a small set of 1-2 parameters, and can be easily parallelised to use multiple CPU cores, giving it major advantages over traditional MCMC algorithms.

\section{Simulations and results}
\subsection{Simulations using GR waveforms}
\subsection{Simulations using modified-GR waveforms}
\section{Conclusions and future work}
%
%
\bibliographystyle{apsrev-nourl}
\bibliography{TGR_HM}

\end{document}
